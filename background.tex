% LAB: Tittel: Background -> Introduction

In recent years the biological community has generated and collected an
unprecedented ammount of data.
While the cost of data collection has drastically decreased, data analysis
continue to be a larger fraction of the total experiment
cost.\cite{sboner2011real}
% LAB: An important part of this cost is the time spent by human experts interpreting the data...
This calls for novel methods in data analysis and
exploration. 

% Trenger litt tekst som sier at vi skal snakke om stats, presentasjon, dbs,
% microservices og kobling av alle. 
Several tools are now available in various programming languages for biological data
analysis. These include novel statistical/bioinformatics methodologies and
graphical analysis tools.
In the R statistical programming language developers share software through
repositories such as CRAN\footnote{\url{cran.r-project.org}} or
Bioconductor\footnote{\url{bioconductor.org}}.  In other languages, libraries
for biological bomputation are often availalbe like BioPython\cite{biopython}
and biogo\cite{biogo} for Python or Go, respectively. 
% LAB: Synes at Galaxy også bør tas med

Although these framework are of tremendous help for scientists with some
computing know-how, even computationally savvy researchers 
% LAB: dette blir vel "muntlig". Jeg synes at dette er stedet å løfte opp utfordringene med å utvikle ordentlig software, dvs:
% 1. Noe som virker
% 2. Som har god nok performance
% 3. Hvor det er mulig etterpå å forstå hva analysen gjorde
% 4. Som skalerer til større datasett
% 5. Som er mulig å vedlikeholde etterpå
% 6. Som har deler som kan gjennbrukes av andre
% Ofte er det bare 1. som blir gjort. Men feks for oss i Elixir som skal tilby tjenester som faktisk virker, så er det de andre punktene som er viktig.
can get tangled up when wrestling with software and big data. 
% LAB: Starte med denne? Og menes det nye apps, eller at Kvik trenges? For sistnevntge så bør det være "novel approach/system"
Novel applications are needed to better integrate statistical
packages, biological databases, and interactive visualizations.
% LAB: Henger ikke helt sammen med forrige setning, og det virker som at det er en gjentagelse av diskusjonen i forrige paragraf. Kanskje best å bare ta bort denne setningen.
Different programming languages solve different tasks. 
For example, new biological data analysis techniques are quickly realeased in R
and its package repositories, high performance computer vision tasks are
performed optimally in C++ and OpenCV, and portable user interfaces more easily
built in HTML, CSS and JavaScript. Therefore applications that integrate novel
statistical analysis tools, interactive visualizations, and biological databases
likely need to include several components written in different languages. 

A microservice architecture structures an application into small reusable,
loosely coupled parts. These communicate via lightweight programming
language-agnostic protocols such as HTTP, making it possible to write single
applications in multiple different programming languages. This way the
most suitable programming language is used for each specific part. To build a
microservice application, developers bundle each service in a container.
Containers are built from configuration files which describe the operating
system, software packages and versions of these. 
% LAB: Unøyaktig. Og det interessante er ikke å reprodusere en applikasjon men en analyse med alle dens parametere, database versjoner, libraries etc
This makes reproducing an
application a trivial task. The most popular implementation of a software
container is Docker\footnote{\url{docker.com}}, but others such as
Rkt\footnote{\url{coreos.com/rkt}} exist. Initiatives such as
BioContainers\footnote{\url{biocontainers.pro}} now provide containers
pre-installed with different bioinformatics tools. 
% LAB: Bør nevne CWL. Mange vil tenke på den som løsningen for det Kvik gjør.

% LAB: Jeg ville hatt requirments innholdet her.

In this paper, we describe a novel approach for building data exploration
applications in systems biology. We show that by building applications as a set
% LAB: "it is possible". Jeg synes ikke vi skal si det slik med mindre det er gjort noe som ikke var mulig før; og da er antageligvis Science rette stedet å sende :)
% LAB: bedre: å si at vi har identifisert services som er: (i) useful for bio data exploration, (ii) reusable, (iii) low overhead/optimized, osv
of services it is possible to reuse and share its components between
applications. In addition, by packaging the services using container technology
we promote reproducible research and simplify application deployment. We have
used our approach to build a number of applications, both command-line and
web-based. In this paper we describe how we used our approach to develop MIxT,
a web application for exploring and comparing transcriptional profiles from
blood and tumor samples. The MIxT web application integrates statistical
analysis together with biological databases and interactive visualizations.

% LAB: Related bio work? Annen microservice related work? Hvorfor ikke bare
% bruke de?

\section*{Requirements} 

% LAB: Jeg ville byttet om paragrafen og listen. Dvs først hva vi har identifisert som requirements etter å ha implementert endel slike apps, og deretter hvordan disse requirements kan implementeres som services som tilbyr abstraksjoner, interfaces, og data strukturer som løser disse requirements.

First, we identified a set of reusable services that application developers can
use to build a wide range of applications. The key services of a biological data
exploration application are i) a compute service for executing statistical
analyses in languages such as R, and ii) a database query service for retrieving
information from biological databases.
% LAB: litt rart at visualization/ GUI ikke er med. Men tanken er kanskje at dette er services som brukes av slike applikasjoner. Isåfall bør det sies.

To build these services we need a framework that fulfills the following
requirements: 

\begin{enumerate}
    \item It provides a language-independent approach for integrating, or
        embedding, statistical software, such as R, directly in 
        % LAB: her er det ikke klart hva som er en slik app? Kanskje bedre å si at det er noe JavaScript. Og at dette er ett eget requirement.
        interactive data
        exploration applications.
    % LAB: denne er misvisende. Jeg tenker enten at dette ikke trenges siden DB allerede tilbyr en interface som kan brukes, eller at det er noe veldig avansert som har laget en unified-interface-to-everything
    % LAB: jeg ville heller sagt at requirment er en interface/wrapper som tilbyr low latency access & provenance
    \item It has an interface to online reference databases to provide meta-data to
        biological entities. % liker ikke entities-ordet, må ha noe annet 
    % LAB: blir kanskje for mye i en bullett? Develop + optimize + maintain i en (har med skriving av kode å gjøre); deploy i en annen; share/ reuse between projects (med share tenker jeg feks en cache delt mellom to applikasjoner).
    \item Its components should be easy to develop, maintain, deploy and share
        between projects. 
    % LAB: Blir feil å si "It scales". Ihvertfall tre type performance utfordringer: (i) response time for interaktive operasjoner, (ii) mulighet til å bruke større dataset, (iii) mange samtidige brukere.
    \item It scales to meet the performance requirements of interactive data
        exploration applications. 
\end{enumerate} 
