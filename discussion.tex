We argue that developing data exploration applications using a microservice
architecture is a viable alternative to the traditional monolithic approach.
By packaging each component of an application in a software container
application developers can reuse and share parts of an application across
research teams and projects. 

Although the partition of components can help break up an application into
manageable parts, there is more overhead with deploying and monitoring these
than a single application. Leslie Lamport's famous quote \emph{You know you have
a distributed system when the crash of a computer you’ve never heard of stops
you from getting any work done.} perfectly describes the possible challenges
application developers face when moving into a microservice architecture.
Monitoring the health of the different services and keeping the services running
is a challenge, but systems such as Kubernetes\footnote{\url{kubernetes.io}}
provide the necessary functionality to manage containerized applications. 

\subsection*{Future work} 
Although we have a first working prototype of the microservices and the MIxT web
application, there are a few points we aim to address in future work.

The first issue is to improve the user experience in the MIxT web application.
Since it is performing many of the analyses on demand, the user interface may
seem unresponsive. We are working on mechanisms that gives the user feedback
when the computations are taking a long time. 

The database service provides a sufficient interface for the
MIxT web application. While we have developed the software packages for
interfacing with more databases, these haven't been included in the database
service yet. In future versions we aim to make the database service be a
interface for all our applications. 
We also aim to improve how we capture data provenance. We aim to provide
database versions and meta-data about when a specific item was retrieved from
the database.


One large concern that we haven't addressed in this paper is security. In
particular one security concern that we plan to address in Kvik is the
restrictions on the execution of code in the compute service. We plan to address
this in the next version of the compute service, using methods such as
AppArmor\footnote{\url{wiki.ubuntu.com/AppArmor}} that can restrict a program's
resource access. 

We also aim to explore different avenues for scaling up the compute service.
Since we already interface with R we can use the Sparklyr or SparkR packages
to run analyses on top of Spark. Using Spark as an execution engine for data
anlyses will enable applications to explore even larger datasets. 
