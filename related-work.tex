In this section we discuss methods for integrating statistical analyses in data
exploration applications, different visualization frameworks, interfaces to
biological reference databases, and methods for containerizing applications. 

\subsection*{Integrate Statistical Analyses} 
Shiny is a web application framework for R\footnote{\url{shiny.rstudio.com}}
It allows developers to build web applications in R without having to have any
knowledge about HTML, CSS or Javascript. 
% LAB: Skrive hvordan og i hva?
% BF: Skjønte ikke helt her? 
Its widget library to provides more advanced Javascript
visualizations such as Leaflet\footnote{\url{leafletjs.com}} for maps or
three.js\footnote{\url{threejs.org}} for WebGL-accellerated
graphics. Developers can choose to host their own web server with the user-built
Shiny Apps, or host them on public servers. Shiny forces users to implement data
exploration applications in R, limiting the functionality to the 
widgets and libraries in Shiny. 

OpenCPU is a system for embedded scientific computing and reproducible
research.\cite{opencpu} Similar to the compute service in Kvik, it offers an
HTTP API to the R programming language to provide an interface with statistical
methods. It allows users to make function calls to any R package and retrieve
the results in a wide variety of formats such as JSON or PDF. 
Users can chose to host their own R server or use public servers, and OpenCPU
works in a single-user setting within an R session, or a multi-user setting
facilitating multiple parallel requests. This makes OpenCPU suitable
for building a service that can run statistical analyses. 
OpenCPU provides a Javascript library for interfacing with R, as well as Docker
containers for easy installation and OpenCPU has been used to build multiple
applications.\footnote{\url{opencpu.org/apps.html}}. 
The compute service in Kvik follows many of the design patterns in
OpenCPU. Both systems interface with R packages using a hybrid state pattern
over HTTP. Both systems provide the same interface to execute analyses and
retrieve results.  While OpenCPU is implemented on top of R and Apache, Kvik is
implemented from the ground up in Go. Because of the similarities in the
interface to R in Kvik we provide packages for interfacing with our own R server
or OpenCPU R servers

Renjin is a JVM-based interpreter for the R programming language.\cite{renjin}
It enables integrating the R interpreter in web applications. Since it is built
on top of the JVM it allows developers to write data exploration applications in
Java that interact directly with R code, both running on top of the JVM.
Although Renjin provides as a service pre-built versions of packages from
CRAN and BioConductor, not all packages can be built for use in Renjin. This
makes it necessary to re-implement the packages that cannot be built for use in
Renjin. 

\subsection*{Parallel and Distributed Execution}
Biogo is a bioinformatics library written in Go. It provides functionality to
analyze genomic and metagenomic datasets in the Go programming
language.\cite{biogo} Using the Go programming language the developers
are able to provide high-performance parallel processing in a clean and simple
programming language. Go provides the ease of programming as popular scripting
languages such as R, but the performance of low-level langauges such as C++. 

MapReduce is a popular framework and programming model to analyze large-scale
datasets using distributed and parallel computing.\cite{dean2008mapreduce}
While MapReduce has shown its usefulness in certain applications, it does not
fit the different statistical methods in bioinformatics. 
An alternative to MapReduce is Spark, which supports high-performance parallel
execution of iterative machine learning methods by partitioning data across a set
of machines and optimizing the data access.\cite{zaharia2010spark} Spark is
growing in popularity, especially with the
Spakrlyr\footnote{\url{spark.rstudio.com}} and
SparkR\footnote{\url{spark.apache.org/docs/latest/sparkr.html}} R packages that
allow R users to run analyses on top of Spark without having to port the
analysis code to another language. 

% Usikker på om ADAM/VariantSpark er noe å ta med. jeg ser ikke heeelt hva de er
% gode for mtp. applikasjonene våre. Føler at Sparklyr dekker Spark-delen. 
ADAM and VariantSpark are systems that are implemented on top of Spark to
analyse genomic data. ADAM provides a set of formats, APIs and processing stage
implementations \cite{massie2013adam} while VariantSpark provides an method for
clustering genomic variants\cite{o2015variantspark}. 

Pachyderm\footnote{\url{pachyderm.io}} is a system for running containerized
data analysis pipelines. Each step in an analysis pipeline is run within a
software container and the output data is version controlled. Pachyderm
automatically partitions and distributes the input data to enable parallel
processing. Pachyderm runs on top of Kubernetes\footnote{\url{kubernetes.io}},
a system for managing and deploying containerized applications. 

\subsection*{Visualization tools} 
Cytoscape is an open source software platform for visualizing complex
networks and integrating these with any type of attribute
data\cite{shannon2003cytoscape}. It allows for analysis and visualization in the
same platform. Users can add additional features, such as databases connections
or new layouts, through Apps. One such app is cyREST which allows external
network creation and analysis through a REST API\cite{ono2015cyrest}.  To bring
the visualization and analysis capabilities to the web the creators of Cytoscape
have developed Cytoscape.js\footnote{\url{js.cytoscapejs.org}}, a Javascript
library to create interactive graph visualizations. 
% LAB: Ulemper?

% Kanskje droppe disse? nevner ikke noe a det de holder på med seinrere... 
Caleydo is a framework for building applications for visualizing and exploring
biomolecular data\cite{cleydo}. Until 2014 it was a standalone tool that needed
to be downloaded, but the Caleydo team are now making the tools web-based. There
have been several applications built using Caleydo: StratomeX for exploring
stratified heterogeneous datasets for disease subtype analysis\cite{stratomex};
Pathfinder for exploring paths in large multivariate graphs\cite{pathfinder};
UpSet to visualize and analyse sets, their intersections and
aggregates\cite{upset}; Entourage and enRoute to explore and visualize
biological pathways \cite{entourage}\cite{enroute}; LineUp to explore rankings
of items based on a set of attributes\cite{lineup}; and Domino for exploring
subsets across multiple tabular datasets\cite{domino}. 

BioJS is an open-source JavaScript framework for biological data
visualization.\cite{gomez2013biojs} It provides a community-driven online
repository with a wide range components for visualizing biological data
contributed by the bioinformatics community. BioJS builds on
node.js\footnote{\url{nodejs.org}} providing both server-side and client-side
libraries. 

% LAB: DB/ provenance related work? Som Galaxy, Pacyderm, CWL, etc

\subsection*{Biological Databases} 
Maybe some words here on how to get data out of the different biological
databases? + possible provenance. 

\subsection*{Containerized analysis} 
In the later years software containers have been widely adopted by both
the software industry as well as research communities. Containers provide an
isolated execution environment that can be used to package and run
an application with all its dependencies, library versions and configuration
files. In researchers containers have become popular because they provides a
reproducible environment that can be shared between research projects. 

In bioinformatics researchers are starting to bundle their software using
software containers, such as Docker. There are repositories such as
BioContainers\cite{dabiocontainers} and BioBoxes\cite{Belmann2015} that provide
containers preinstalled with software for doing analyses and running different
applications. Systems such as Galaxy now allows researchers to build analysis
pipelines where each step is executed within a software container. 

% Pipeline, distributed exec, provenance. 

\subsection*{Kvik and Kvik Pathwys}
We have previously built a system for interactively exploring gene expression
data in context of  biological pathways.\cite{fjukstad2015kvik} Kvik Pathways is
a web application that integrates gene expression data from the Norwegian Women
and Cancer (NOWAC) cohort together with pathway images from the Kyoto
Encyclopedia of Genes and Genomes (KEGG). We used the experience building Kvik
Pathways to completely re-design and re-implement
the R interface in Kvik. From having an R server that can run a set of functions
from an R script, it now has a clean interface to call any function from any R
package, not just retrieving data as a text string but in a wide range of
formats. We also re-built the database interface, which is now a seperate
service. This makes it possible to leverage its caching capabilities to improve
latency. This transformed the application from being a single monolithic
application into a system that
consists of a web application for visualizing biological pathways, a database
service to retrieve pathway images and other metadata, and a compute service for
interfacing with the gene expression data in the NOWAC cohort. We could then
re-use the database and the compute service in the MIxT application. 

