% 1M skyldes lønn til mennesker eller kun compute? (tid = penger hvis det er
% mennesker)

% BF: Har du en referanse vi kan bruke her? Dette er en ganske vanlig quote, men
% husker ikke hvor jeg har sett den, Dette er veldig generelle greier, men vi
% skulle hatt et par gode paper å sitere. 
% LAB: Rob Finn (EBI) bruker denne: https://genomebiology.biomedcentral.com/articles/10.1186/gb-2011-12-8-125
% LAB: Også har vi disse: https://www.veritasgenetics.com/mygenome
In recent years the collection of biological data and curation of biological
datasets has been unprecedented. While the cost of collecting data has
drastically decreased, data analysis continue to be a larger fraction of the
total experiment cost.\cite{sboner2011real} This calls for novel methods in data
analysis and exploration. 

% Trenger litt tekst som sier at vi skal snakke om stats, presentasjon, dbs,
% microservices og kobling av alle. 
To explore the growing number of biological data sets, there are now a number of
analysis tools in various programming languages. There are
both new methods for analyzing the data as well as presenting them using novel
visualization techniques. 
In the R statistical programming language developers can share software
packagesr for exploring high-throughput omics datasets through repositories such
as CRAN\footnote{\url{cran.r-project.org}} or
Bioconductor\footnote{\url{bioconductor.org}}.
In other languages such as Python or Go, developers can choose bioinformatics
libraries such as BioPython\cite{biopython} and biogo\cite{biogo} respectively.
Such frameworks provide functionality for analyzing data, linking to databases
and visualizing the data. 

While these frameworks require its users to be proeficient at coding, there is a
need for applications that allow researchers to explore datasets interactively
through simple user interfaces. These applications need to integrate statistical
packages, biological databases and interactive visualizations. Unarguably
different programming languages are suitable for solving different tasks. E.g.
to use R and Bioconductor to analyse biological data, or C++ and OpenCV for
optimized high-performance computer vision tasks, or HTML, CSS, and JavaScript
to build portable user-interfaces. We argue that to build applications that
integrate statistical analyses, interactive visualizations, and biological
databases it is reasonable to compose the application of several components
written in different languages. 

A microservice architecture structures an application into small reusable,
loosely coupled parts. These communicate via lightweight programming
language-agnostic protocols such as HTTP, making it possible to write single
applications in multiple different programming languages. This makes it possible
to use the most suitable programming language for each specific part. To build a
microservice application, developers bundle each service in a container that are
deployed. Containers are built from configuration files which describe the
operating system, software packages and versions of these. This makes
reproducing an application a trivial task. The most popular implementation of a
software container is Docker\footnote{\url{docker.com}}, but others such as
Rkt\footnote{\url{coreos.com/rkt}} exist.  


% Traditionally data exploration applications have been built as desktop
% applications that require installation and setup by the user, but now the move
% is towards applications that run in the web browser without any user setup.  
% Examples of such tools are the Integrative Genomics
% Viewier (IVG)\cite{thorvaldsdottir2013integrative} for  
% WIP WIP WIP
% In systems biology there is a need for visualization tools that can make use
% of the lastest statistical packages and interact with these. It is not enough
% to visualize the results of an analysis pipeline, but interact with the
% different steps from the final visualization. Examples of this interaction can
% be modifying clustering parameters that are used in the analysis.

% % Data fra R til vis
% To visualize and share results from statistical analyses, the results are often
% exported to a data format such as comma-separated values (CSV) and
% then visualized using an external tool. This decouples data presentation and
% analysis.  
% % LAB: Hva er bra/dårlig med decoupling
% Through initiatives such as ApacheR and OpenCPU there has been a move
% towards embedding scientific computation in data visualization application. This
% removes the decoupling of statistical frameworks and interaction with the
% analysis results. 
% 
% % Koble data sammen med db. 
% % LAB: Hva er sammenhengen med programming packages & viz tools?
% Interpreting the analysis results require integration of known biology, either
% from biological databases such as MSigDB\cite{} or KEGG\cite{}, or through
% scientific publications from reference databases such as PubMed\cite{}.
% Performing manual lookup into these databases is often tedious and error-prone,
% making it necessary to automate the task. Now as more databases provide REST
% APIs it is possible to provide software packages for automatic retrival of
% database information. In addition, since databases are continuously being
% updated, using a REST API to retrieve database information will ensure that the
% data is always kept up to date. 

% Hvorfor bryr vi oss om både stats + DBs?  Skal vi si noe om at det blir
% vanligere å bruke REST APIs fremfor å laste ned alt og lese ting lokalt
% Lage apps som gjør alt dette. 
% Hva er micro-services? 
% LAB: Hadde forventet en beskrivelse av ett problem (motivert i paragrafene
% overnfor), som microservices løser


% Har sett et par (bla alex lex sin siste og en VR-artikkel fra i fjor) at det 
% er vanlig med et avsnitt på slutten av bg/intro som beskriver hva artikkelen
% handler om. 
In this paper we describe a novel approach for building data exploration
applications in systems biology. We show that by building applications as a set
of services it is possible to reuse and share its components between
applications. In addition, by packaging the services using container technology
we promote reproducible research and simplify application deployment. We have
used our approach to build a number of applications, both command-line and
web-based. In this paper we describe how we used our approach to develop MIxT,
a web application for exploring and comparing transcriptional profiles from
blood and tumor samples. The MIxT web application integrates statistical
analysis together with biological databases and interactive visualizations.

% LAB: Related bio work? Annen microservice related work? Hvorfor ikke bare
% bruke de?

\section*{Requirements} 
From our experience building data exploration applications we have identified a
set of reusable services that application developers can use to build a wide
range of applications. The key services of a biological data exploration
application are i) a compute service for executing statistical analyses in
languages such as R, and ii) a database query service for retrieving information
from biological databases.
On top of these services is possible to build any number applications and these
can be reused by different applications. 

To build these services we need a framework that fulfills the following
requirements: 

\begin{enumerate}
    \item It provides a language-independent approach for integrating, or
        embedding, statistical software, such as R, directly in interactive data
        exploration applications.
    \item It provides an interface to online databases to provide meta-data to
        biological entities. % liker ikke entities-ordet, må ha noe annet 
    \item Its components should be easy to develop, maintain, deploy and share
        between projects. 
\end{enumerate} 


% \subsubsection*{Contributions} 
% Our contributions are: 
% \begin{enumerate}
% % men ikke visualization?
% \item An approach for developing data exploration applications for systems
% biology that combine statistical analyses with online databases.  
% \item A demonstration of its viability through N different applications. 
% \item Performance evaluation of its central data analysis component. 
% \end{enumerate} 

