%% BioMed_Central_Tex_Template_v1.06
%%                                      %
%  bmc_article.tex            ver: 1.06 %
%                                       %

%%IMPORTANT: do not delete the first line of this template

%%%%%%%%%%%%%%%%%%%%%%%%%%%%%%%%%%%%%%%%%
%%                                     %%
%%  LaTeX template for BioVis 2014  %%
%%      article submissions     %%
%%          adapted from BMC    %%
%%          <8 Jan 2014>              %%
%% Liz Marai (g.elisabeta.marai@gmail.com) %%
%%                                     %%
%%%%%%%%%%%%%%%%%%%%%%%%%%%%%%%%%%%%%%%%%


%%%%%%%%%%%%%%%%%%%%%%%%%%%%%%%%%%%%%%%%%%%%%%%%%%%%%%%%%%%%%%%%%%%%%
%%                                                                 %%
%%%%%%%%%%%%%%%%%%%%%%%%%%%%%%%%%%%%%%%%%%%%%%%%%%%%%%%%%%%%%%%%%%%%%

%%% additional documentclass options:
%  [doublespacing]
%  [linenumbers]   - put the line numbers on margins

%%% loading packages, author definitions

\documentclass[twocolumn]{bmcart}% uncomment this for twocolumn layout 



%%% Load packages
%\usepackage{amsthm,amsmath}
%\RequirePackage{natbib}
%\RequirePackage{hyperref}
\usepackage[utf8]{inputenc} %unicode support
%\usepackage[applemac]{inputenc} %applemac support if unicode package fails
%\usepackage[latin1]{inputenc} %UNIX support if unicode package fails
\usepackage{graphicx}

%\def\includegraphic{}
%\def\includegraphics{}

%%% Put your definitions there:
\startlocaldefs
\endlocaldefs

%%% Begin ...
\begin{document}
%%% Start of article front matter
\begin{frontmatter}
\begin{fmbox}
\dochead{Research}

\title{Building systems for interactive data exploration in systems biology}

\author[
     addressref={aff1},
  % noteref={n2},
   email={bjorn.fjukstad@uit.no}
]{\inits{AA}\fnm{Anonymous} \snm{Author1}}
\author[
   addressref={aff1},                   % id's of addresses, e.g. {aff1,aff2}
   %corref={aff1},                       % id of corresponding address, if any
   %noteref={n1},                        % id's of article notes, if any
  email={alsoanonymous@gmail.com}   % email address
]{\inits{AA2}\fnm{Anonymous} \snm{Author2}}

%%%%%%%%%%%%%%%%%%%%%%%%%%%%%%%%%%%%%%%%%%%%%%
%%                                          %%
%% Enter the authors' addresses here        %%
%%                                          %%
%% Repeat \address commands as much as      %%
%% required.                                %%
%%                                          %%
%%%%%%%%%%%%%%%%%%%%%%%%%%%%%%%%%%%%%%%%%%%%%%

\address[id=aff1]{%                           % unique id
  \orgname{UiT – The Arctic University of Tromsø}, % university, etc
  \postcode{9037}                                % post or zip code
  \city{Tromsø},                              % city
  \cny{NO}                                    % country
}
\address[id=aff2]{%                           % unique id
  \orgname{Anonymous Also}, % university, etc
%  \street{210 South Bouquet},                     %
  \postcode{15260}                                % post or zip code
  \city{Williamsburg},                              % city
  \cny{UK}                                    % country
}


%%%%%%%%%%%%%%%%%%%%%%%%%%%%%%%%%%%%%%%%%%%%%%
%%                                          %%
%% Enter short notes here                   %%
%%                                          %%
%% Short notes will be after addresses      %%
%% on first page.                           %%
%%                                          %%
%%%%%%%%%%%%%%%%%%%%%%%%%%%%%%%%%%%%%%%%%%%%%%

\begin{artnotes}
%\note{Sample of title note}     % note to the article
%\note[id=n1]{Equal contributor} % note, connected to author
%\note[id=n2]{Equal contributor} % note, connected to author
%\note[id=n3]{Equal contributor} % note, connected to author
%\note[id=n4]{Project leader and equal contributor} % note, connected to author
\end{artnotes}

\end{fmbox}% comment this for two column layout

%%%%%%%%%%%%%%%%%%%%%%%%%%%%%%%%%%%%%%%%%%%%%%
%%                                          %%
%% The Abstract begins here                 %%
%%                                          %%
%% Please refer to the Instructions for     %%
%% authors on http://www.biomedcentral.com  %%
%% and include the section headings         %%
%% accordingly for your article type.       %%
%%                                          %%
%%%%%%%%%%%%%%%%%%%%%%%%%%%%%%%%%%%%%%%%%%%%%%

\begin{abstractbox}

\begin{abstract} % abstract, must be under 350 words

\parttitle{Background} 
\parttitle{Results} 
\parttitle{Conclusions}  

\end{abstract}

%%%%%%%%%%%%%%%%%%%%%%%%%%%%%%%%%%%%%%%%%%%%%%
%%                                          %%
%% The keywords begin here                  %%
%%                                          %%
%% Put each keyword in separate \kwd{}.     %%
%%                                          %%
%%%%%%%%%%%%%%%%%%%%%%%%%%%%%%%%%%%%%%%%%%%%%%

\begin{keyword}
\kwd{Molecular Sequence Analysis}
\kwd{Molecular Structural Biology}
\kwd{Computational Proteomics}
\end{keyword}

\end{abstractbox}
%
%\end{fmbox}% uncomment this for twcolumn layout

\end{frontmatter}

\section*{Background}

\section*{Methods}

\subsection*{My Subsection}

\begin{figure*}[t!]
  \centering
  \includegraphics[height=3.5in]{placeholder.eps}
  \caption{Example two-column figure. To insert a single column figure, remove the * in the two figure tags. }
  \label{fig:interface}
\end{figure*}


\section*{Results and Discussion}

\section*{Conclusions}


\begin{backmatter}

\section*{List of abbreviations used}
TIM: \textit{triosephosphate isomerase}, scTIM: \textit{saccharomyces
cerevisiae triosephosphate isomerase}, dTIM: \textit{defective triosephosphate
isomerase}, PDB: \textit{protein data bank}.

\section*{Competing interests}
The authors declare that they have no competing interests.

% if your bibliography is in bibtex format, use those commands:
\bibliographystyle{bmc-mathphys} % Style BST file
\bibliography{paper}      % Bibliography file (usually '*.bib' )


%%%%%%%%%%%%%%%%%%%%%%%%%%%%%%%%%%%
%%                               %%
%% Tables                        %%
%%                               %%
%%%%%%%%%%%%%%%%%%%%%%%%%%%%%%%%%%%

%% Use of \listoftables is discouraged.
%%
%\section*{Tables}
%\begin{table}[h!]
%\caption{Sample table title. This is where the description of the table should go.}
 %     \begin{tabular}{cccc}
%        \hline
 %          & B1  &B2   & B3\\ \hline
 %       A1 & 0.1 & 0.2 & 0.3\\
 %       A2 & ... & ..  & .\\
 %       A3 & ..  & .   & .\\ \hline
 %     \end{tabular}
%\end{table}

%%%%%%%%%%%%%%%%%%%%%%%%%%%%%%%%%%%
%%                               %%
%% Additional Files              %%
%%                               %%
%%%%%%%%%%%%%%%%%%%%%%%%%%%%%%%%%%%

%\section*{Additional Files}
%  \subsection*{Additional file 1 --- Sample additional file title}
%    Additional file descriptions text (including details of how to
%    view the file, if it is in a non-standard format or the file extension).  This might
%    refer to a multi-page table or a figure.

%  \subsection*{Additional file 2 --- Sample additional file title}
%    Additional file descriptions text.

\end{backmatter}
\end{document}
