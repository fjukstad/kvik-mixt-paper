\begin{abstractbox}
    \begin{abstract} % abstract, must be under 350 words
        \parttitle{Background} 
            In scientific fields such as systems biology there is a need for
            interactive data exploration tools to enable new insights in the
            fast growing datasets. 
            These tools combine advanced
            statistical analyses, known biology from up-to-date databases, and
            interactive visualizations. The problem solved by each tool differs,
            but typically their underlying components are similiar.  
            % Skal vi si "can" eller "should"?  Kan kanskje også legge inn noen
            % ord om hvorfor dette er bedre, slik at det er helt klart for
            % leseren.
            Application developers can therefore compose applications of
            reusable services rather than implementing a single monolithic
            application. 
            
        \parttitle{Results} 
            We have designed an approach for developing data exploration
            applications in systems biology that builds on the microservice
            architecture. We use this approach to build applications that
            integrate advanced statistical software and up-to-date information
            from biological databases.
            We demonstrate its viability through a web application for exploring
            and comparing transcriptional profiles from blood and tumor samples.
            In addition we have used it to build two other web-applications and
            several command-line tools.  With a microservices approach we can
            re-use and share key components between application reducing
            development, deployment and maintenance time. 
            % LAB: noe om performance & noe om code quality?

        \parttitle{Conclusions}  
            Our approach and reference implementation Kvik is open-sourced at
            \url{github.com/fjukstad/kvik} and the web application for exploring
            transcriptional profiles, MIxT, is availible at
            \url{github.com/fjukstad/mixt}. 

    \end{abstract}
    
    \begin{keyword}
    % \kwd{Molecular Sequence Analysis}
    \end{keyword}
\end{abstractbox}

