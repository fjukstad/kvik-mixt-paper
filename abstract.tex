\begin{abstractbox}
    \begin{abstract} % abstract, must be under 350 words
        \parttitle{Background} 
            With the advent of high-throughput genomics, there is a growing need for
            interactive data exploration tools as the systems biology community
            collects data at an unprecedented rate. 
            These tools need to combine advanced statistical analyses, relevant
            prior knowledge, and interactive visualizations in an applicaiton
            with simple user interfaces.
            Although data analyses, user interfaces and visualizations are
            specialized, application developers can share and reuse the
            underlying components of the application.
            To answer specific research questions tools must
            provide specialized user interfaces and visualizations. While these
            are application-specific, the underlying components of a
            data analysis tool can be shared and reused later. 
            Application developers can therefore compose applications of
            reusable services rather than implementing a single monolithic
            application. 
            
        \parttitle{Results} 
            Our approach for developing data exploration
            applications in systems biology builds on the microservice
            architecture. The resulting applications
            integrate advanced statistical software, up-to-date information
            from biological databases and modern data visualization libaries. 
            We demonstrate the viability through the  MIxT Blood-Tumor web
            application that explore 
            and compare transcriptional profiles from blood and tumor samples in
            breast cancer patients. This approach was reused in two other web
            applications and several command-line tools. 
            % LAB: Blir litt gjentagende her.
            Thus our microservice approach building on software container
            technology enables re-use and sharing of key components between
            application reducing development, deployment and maintenance time. 
            % LAB: noe om performance & noe om code quality?

        \parttitle{Conclusions}  
            Our approach and reference implementation Kvik is open-sourced at
            \url{github.com/fjukstad/kvik}. The web application for exploring
            transcriptional profiles, MIxT, is available at
            \url{mixt-blood-tumor.bci.mcgill.ca} and its source code at 
            \url{github.com/fjukstad/mixt}. 

    \end{abstract}
    
    \begin{keyword}
    % \kwd{Molecular Sequence Analysis}
    \end{keyword}   
\end{abstractbox}

