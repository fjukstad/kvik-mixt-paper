\begin{abstractbox}
    \begin{abstract} % abstract, must be under 350 words
        \parttitle{Background} 
            In scientific fields such as systems biology there is a need for
            interactive data exploration tools to enable new insights in the
            fast growing datasets. 
            These tools combine advanced
            statistical analyses, known biology from up-to-date databases, and visualizations.
            The problem solved by each tool differs, but typically their underlying
            components stay the same.  
            % Skal vi si "can" eller "should"?
            % Kan kanskje også legge inn noen ord om hvorfor dette er bedre, slik at det er helt klart for leseren.
            Application developers can therefore
            compose applications of reusable services rather than implementing a
            single monolithic application.
            
        \parttitle{Results} 
            We have designed an approach for developing data exploration
            applications in systems biology that builds on the microservice
            architecture which 
            % Gjentar det som står ovenfor, kan heller si hva slags services her (R, ...)
            structures an application as a collection of
            services rather than a large monolithic entity. 
            % Litt svakt med kun en applikasjon. Kan vi ikke ta med de eldre også?
            We have demonstrated its viability
            through a web application for exploring and comparing
            transcriptional profiles from blood and tumor samples. 
            % "We were able" og "language-agnostic" er ett svakt/kjedelig resultat. Hva er fordelen som applikasjon utvikler? Hva er fordelen for sluttbrukerene? Er det fordeler for sysadmins?
            Using this approach we are able to build language-agnostic data
            exploration applications that interface with advanced statistical
            software and up-to-date information from biological databases.  

        \parttitle{Conclusions}  
            % Gjentar det som står i Results
            Our approach and reference implementation Kvik, enables easy
            development of data exploration tools that interface with
            statistical analyzes and online databases. 
            Kvik is open-sourced at \url{github.com/fjukstad/kvik} and the web
            application for exploring transcriptional profiles, MIxT, is 
            availible at \url{github.com/fjukstad/mixt}. 

    \end{abstract}
    
    \begin{keyword}
    % \kwd{Molecular Sequence Analysis}
    \end{keyword}
\end{abstractbox}

