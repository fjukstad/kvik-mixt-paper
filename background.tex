In recent years the biological community has generated and collected an
unprecedented ammount of data.
While the cost of data collection has drastically decreased, data analysis
continue to be a larger fraction of the total experiment
cost.\cite{sboner2011real} This calls for novel methods in data analysis and
exploration. 

% Trenger litt tekst som sier at vi skal snakke om stats, presentasjon, dbs,
% microservices og kobling av alle. 
Several tools are now available i various programming languages for data
analysis. These include novel statistical/bioinformatics methodologies and
graphical analysis tools.
In the R statistical programming language developers share software through
repositories such as CRAN\footnote{\url{cran.r-project.org}} or
Bioconductor\footnote{\url{bioconductor.org}}.  In other languages, libraries
for biological bomputation are often availalbe like BioPython\cite{biopython}
and biogo\cite{biogo} written in Python or Go, respectively. 

Although these framework are of tremendous help for scientists with some
computing know-how, even computationally savvy researchers can get tangled up
when wrestling with software and big data. 
Novel applications are needed to better integrate statistical
packages, biological databases, and interactive visualizations.
Different programming languages solve different tasks. 
For example, new biological data analysis techniques are quickly realeased in R
and its package repositories, high performance computer vision tasks are
performed optimally in C++ and OpenCV, and portable user interfaces more easily
built in HTML, CSS and JavaScript. Therefore applications that integrate novel
statistical analysis tools, interactive visualizations, and biological databases
likely need to include several components written in different languages. 

A microservice architecture structures an application into small reusable,
loosely coupled parts. These communicate via lightweight programming
language-agnostic protocols such as HTTP, making it possible to write single
applications in multiple different programming languages. This way the
most suitable programming language is used for each specific part. To build a
microservice application, developers bundle each service in a container.
Containers are built from configuration files which describe the operating
system, software packages and versions of these. This makes reproducing an
application a trivial task. The most popular implementation of a software
container is Docker\footnote{\url{docker.com}}, but others such as
Rkt\footnote{\url{coreos.com/rkt}} exist. Initiatives such as
BioContainers\footnote{\url{biocontainers.pro}} now provide containers
pre-installed with different bioinformatics tools. 
% LAB: Kan nevne CWL 

In this paper, we describe a novel approach for building data exploration
applications in systems biology. We show that by building applications as a set
of services it is possible to reuse and share its components between
applications. In addition, by packaging the services using container technology
we promote reproducible research and simplify application deployment. We have
used our approach to build a number of applications, both command-line and
web-based. In this paper we describe how we used our approach to develop MIxT,
a web application for exploring and comparing transcriptional profiles from
blood and tumor samples. The MIxT web application integrates statistical
analysis together with biological databases and interactive visualizations.

% LAB: Related bio work? Annen microservice related work? Hvorfor ikke bare
% bruke de?

\section*{Requirements} 
First, we identified a set of reusable services that application developers can
use to build a wide range of applications. The key services of a biological data
exploration application are i) a compute service for executing statistical
analyses in languages such as R, and ii) a database query service for retrieving
information from biological databases.

To build these services we need a framework that fulfills the following
requirements: 

\begin{enumerate}
    \item It provides a language-independent approach for integrating, or
        embedding, statistical software, such as R, directly in interactive data
        exploration applications.
    \item It has an interface to online reference databases to provide meta-data to
        biological entities. % liker ikke entities-ordet, må ha noe annet 
    \item Its components should be easy to develop, maintain, deploy and share
        between projects. 
    \item It scales to meet the performance requirements of interactive data
        exploration applications. 
\end{enumerate} 
