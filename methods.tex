In this section we first motivate our microservice approach based on our
experiences developing the MIxT web application.
We describe the process from initial data analysis to the final application,
highlighting the importance
% LAB + performance/scalability/deployment (dvs noe som er kvantifiserbart)
of language-agnostic services to facilitate the use of different tools in
different parts of the application. 
%This is especially important in
%interdisciplinary teams where researchers use a wide range of tools and
%programming languages.  
%We believe many systems biology data exploration applications are developed
%similarly and that they can therefore also benefit from the microservice
%approach. 
Based on our experiences, we then generalize the ideas to a set
of principles and services that can be reused and shared between applications. 
% LAB: + Design and impl of micro-services.

% Bruke MIxT til å fortelle hvordan de forskjellige delene ble gjort og hvordan
% man kan bygge en app uten å gjore alt på nytt igjen, eller hvertfall kunne
% bygge på det man har. 
% 1) les inn data (bioconductor) 
% 2) se på de slå sammen, kanskje enkle vis
% 3) gjøre analyser

% 4) koble resultat opp mot db
% 5) gå gjennom resultatene og db interaktivt.

% 6) del med andre 

\subsection*{MIxT} 
We analyzed profiled RNA in blood and matched tumor from 173 patients in the
Norwegian Women and Cancer (NOWAC) study. Each profile measures the expression
of 16 782 unique genes. We used Weighted Gene Correlation Network Analysis
(WGCNA)\cite{langfelder2008wgcna} to cluster the genes in each tissue
based on co-expression. From these clusters of genes, called modules, we
investigated their relationship to known biological processes.
\emph{More on the methods maybe??}

From the analyses we built an R package\cite{mixt-r-package} that
implements the different statistical methods developed in the MIxT project. 
The R package contains functionality for running statistical analyses and
generating specialized static visualizations. Based on the results and analyses
we wanted to build an application for quickly exploring the results. Since we
have a large code base already developed in R we needed a system that can
directly interface with the R package. 
In addition, the interface to R should be accessible through standard protocols,
such as HTTP, not enforcing any programming language or platform on the
application developer. 

A large part of biological data research is to link the results to known biology
from literature or reference databases. In the MIxT project we needed an
application that could interface with a set of different databases, keeping the
information up-to-date. As with the R interface the interface to the databases
should be accessible from any programming language. 

A key feature that motivated the design of MIxT was that we want to
have the flexibility to run the application on any platform. Bundling each
component in a software container such as Docker allows us to deploy the
application on a wide range of hardware, from local installations to deployments
to cloud providers such as Amazon Web Services\footnote{\url{aws.amazon.com}}.


% BF: Hei LA var dette noe sånn du så for deg? 
% LAB: Generelt; 1. Hva var problemet? Hva ble gjort? Hvordan?
% LAB: Så for R: 1. Hva slags analyse skal gjøres?; det ble vel utviklet nye
% statistiske/ bioinformatikk metoder (som er tidligere publisert?); R fordi det
% er velegnet for å utvikle nye metoder...
% LAB: Deretter: visualisering/ web app... 
% LAB: Også: integrasjon av visualisering/ web app + R
% LAB: integrasjon mot databaser
% LAB: Tilslutt: hvordan den er deployet.
% Generealisering kan gjøres i neste avsnitt der microservice apporach
% blir beskrevet. Også ville jeg likt å se tall her. Hvor stort er datasettet?
% Hvor lang tid tar det? Er dette i critial path for data exploration eller bare
% pre-processing? 
% Hvordan gjøres dette? Er dette egentlig protptyping, og dermed en slags
% requirements spec for de endelige visualiseringene? Kan dette gjenbrukes eller
% er det mest bruk-og-kast? Hvem gjør dette? 
% Database lookup for hva?
% After this analysis we often end up with genes or lists of genes of
% interest that we can use to guide database lookup.
% Savner en diskusjon om performance, resource usage, og andre krav til MiXT
% Hva med non-quirky visualiseringer integrert med database lookup?
% Hva med visualiseringer for andre brukere? Og data?
% Hva med neste verktøy, dvs MIxT for eksempel for methylation + gene
% expression?
% Hva ville vi egentlig med interaktive visualiseringer? Hvordan burde det gjøres? 
% movivasjon! 
% How did we go a head with the MIxT app? What did we consider etc? Words phrases
% etc. 


% Noe av dette er kanskje allerede i "Building applications"
% BF: One key point: We can reuse building blocks such as the engine for
% executing statistical analyses or the REST api to get stuff from databases.
% What we can't re-use is the application logic and application specific
% visualizations. Sure we can reuse heatmaps or barcharts etc, but they will
% most likely be application specific. 
\subsection*{Kvik}
Based on the development of MIxT, we generalized its functionality and
underlying systems into a set of design principle guidelines for application
developers: 

\textbf{Principle 1}: Build applications as collections of language-agnostic
microservices. This makes it possible to re-use key components and build
specialized data exploration applications in the most suitable programming
language. 

\textbf{Principle 2}: Deploy each service using container technology such as
Docker. This has a number of benefits. It simplifies deployment itself, it makes
it trivial to share services between projects and research groups, and it
ensures reproducible services.

\textbf{Principle 3}: Package statistical methods and data as software packages
that can be used by power-users and the data exploration tools themselves. An
example is to build an application using R packages and OpenCPU or Kvik. This
makes it possible to either explore the data and methods through the data
exploration application or an R session. 

From these three main principles we built a set of software packages that
provide functionality for microservices that can be used to build a data
exploration application in systems biology. We built a compute service for
executing statistical analyses and a database service that provides data from
biological databases. Using these it is possible to develop specialized data
exploration application in any modern programming language.  

% Også er det viktig å ikke glemme "system aspects" performance, management,
% deployment, etc for disse. Det kn enten forklares her eller senere.
% Det er ikke forklart hvordan ting henger sammen i Kvik, så dette er vanskelig
% å forstå
% LAB: her er stedet for alle Go bibliotek og andre lavnivå detlajer

\subsubsection*{Compute Service}
% Describe how we've designed the interface with R: Build an R-package and call
% functions from it, we provide four different output formats to the user (json,
% csv, pdf, png),  as well as four different http endpoints (call, get and rpc).

% Hva er fordelene med å gjøre det i go?
The compute service in Kvik is built using a hybrid state
pattern\cite{opencpu}.
We provide three main operations for interfacing with R:
Call, Get, and RPC. The Call operation is used to execute and run a function
from an R package. It takes as input an R package name, a function name, and
optional arguments. It returns a unique identifier that later can be used by the
Get operation to retrieve results. The Get operation is used to get results in
different output formats, e.g. JSON, CSV, PDF, or PNG. The RPC is just a
combination of a Call and a subsequent Get. 

% With this process in mind, we designed the interface to the R programming
% language in Kvik. We want to make it possible to call any function from an R
% package and return its results either as plain text, such as comma-separated
% tables, or binaries such as images. Enforcing that R code is built into R
% packages ensures that the analysis code can be used by power users through an
% ordinary R session or in the data exploration application itself. 



\subsubsection*{Database Service} 
The database service provides an interface to biological databases for
retrieving meta-data on genes and processes. In Kvik we have built packages for
interfacing with The Entrez Programming Utilities
(E-utilities)\footnote{\url{eutils.ncbi.nlm.nih.gov}},
MSigDB\footnote{\url{software.broadinstitute.org/gsea/msigdb}}, Hugo Gene
Nomenclature Committe (HGNC)\footnote{\url{genenames.org}}, and Kyoto Encyclopedia
of Genes and Genomes (KEGG)\footnote{\url{kegg.jp}}

The database service uses a caching mechanism to reduce the load on the online
databases. It will also speed up subsequent queries for a cached object, since
the query can be served out of cache and not having to be fetched from a remote
database. We allow application developers to specify the cache eviction policy,
but on default the database service does not evict anything from its cache
before the service is restarted. This can be modified by the application
developer. 


% lokalt: fortere, kan cache, kan få bort last fra de sentrale serverne 
% Kan også beskrive hva slags interfaces de eskponerer, hva som er performance
% characteristics, programmerings utforderinger, og tilatt bruk
% TODO: Describe the interfaces/API. 
% + abstraksjoner som tar seg av caching og provenance management
% Jeg ser for meg at det er nyttig å kunne si for alle database oppslag noe sånt
% som: read cached value = False, cache result = "session". Dvs alltid les
% nyeste verdi, men cache resultatene for denne session. Kanskje er det også
% andre database-generisk operasjoner som er nyttige abstraksjoner (hent alle
% entries i en liste). Også er det sikkert mulig å pakke disse inn i en
% interface som kan brukes til å implementere database spesifike (KEGG, MsigDB)
% komponenter.


\subsubsection*{Building applications} 
In Kvik we use R packages as the fundamental building block for data exploration
applications. They provide an interface to data and analyses, and especially in
the field of systems biology, the R programming language provide the largest
collection of data analysis packages. % litt vagt kanskje? 

An application starts as one or more R packages with the datasets, analysis
code, and optional other utilities for generating static plots in R.  Developers
can then run the compute service which makes the functions from the R package
accessible through an HTTP endpoint. With the data analysis and resulting
datasets available, developers can focus on 

It is important to note that since the end application interfaces directly with
R, developers can leverage this to produce dynamic visualizations. For example,
if an application uses a clustering method to color nodes in a graph, end-users
can tweak parameters 

In Kvik we do not specify what programming language or set of tools to use to
build an application. We do not believe that there is any one language or
framework suitable for every application. We believe that by orchestrating an
application as a set of services communicating over standard protocols 


\subsection*{Implementation}
In ths section we describe the implementation details in Kvik and the
microservices required to build the MIxT web application. 

Kvik is implemented as a collection of Go packages with the
functionality required to build services that can integrate statistical
software in a data exploration and provide an interface to up-to-date biological
databases. We chose the Go programming language because of its performance, ease
of development, and simple deployment. 
To integrate R we provide two packages \emph{gopencpu} and
\emph{r}, that interface with OpenCPU and Kvik R servers respectively. To
interface with biological databases we provide the packages \emph{eutils},
\emph{gsea}, \emph{genenames}, and \emph{kegg} that interface with E-utils,
MsigDB, HGNC and KEGG respectively.
In addition to these packages we provide Docker images that implement the
two required microservices. 

Both the compute and the databases service in Kvik builds on the standard
\emph{http} library in Go. On start the compute service 
launches a user-defined number of R sessions that execute analyses on demand.
This allows for parallel execution of analyses. We provide a simple FIFO queue
for queuing of requests. The compute service also provides the opportunity for users to
cache analysis results to speed up subsequent calls. The database service use
the \emph{gocache}\footnote{\url{github.com/fjukstad/gocache}} package to cache
any query to an online database.

% LAB: Litt usikker på om dette hører til i Results eller Methods
% \subsection*{Applications}
% Stress.
% Pathways.
% MIxT .
% Command line-man. 
% 
% % LAB: kort beskrivelse av hva alle apps gjør
% 
% % LAB: Figur som viser hva som er felles og ulikt for alle appene. Her bør noe
% % være likt ellers har vi bare 3-4 applikasjoner :)
% 
% % LAB: mer detaljert beskrivelse av hvordan hver app er implementert med Kvik
% 
% 

