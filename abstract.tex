\begin{abstract} % abstract, must be under 350 words
        As the systems biology community generates and collects data at an
        unprecedented rate, there is a growing need for interactive data
        exploration tools to explore the datasets. 
        These
        tools need to combine advanced statistical analyses, relevant knowledge
        from biological databases, and interactive visualizations in an
        application with clear user interfaces.  To answer specific research
        questions tools must provide specialized user interfaces and
        visualizations. While these are application-specific, the underlying
        components of a data analysis tool can be shared and reused later.
        Application developers can therefore compose applications of reusable
        services rather than implementing a single monolithic application from
        the ground up for each project. 
        
        Our approach for developing data exploration applications in systems
        biology builds on the microservice architecture. Microservice
        architectures separates an application into smaller components that
        communicate using language-agnostic protocols. We show that this design
        is suitable in bioinformatics applications where applications often use
        different tools, written in different languages, by different research
        groups. Packaging each service in a software container  enables
        re-use and sharing of key components between applications, reducing
        development, deployment, and maintenance time. 

        We demonstrate the viability of our approach through a web application,
        MIxT blood-tumor, for exploring and comparing transcriptional profiles
        from blood and tumor samples in breast cancer patients. The application
        integrates advanced statistical software, up-to-date
        information from biological databases, and modern data visualization
        libraries. 

        The web application for exploring
        transcriptional profiles, MIxT, is online at
        \url{mixt-blood-tumor.bci.mcgill.ca} and open-sourced at 
        \url{github.com/fjukstad/mixt}. Packages to build the supporting
        microservices are open-sourced as a part of Kvik at
        \url{github.com/fjukstad/kvik}. 

\end{abstract}

