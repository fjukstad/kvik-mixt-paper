% Hva er korrekt: "in" system biology eller "for" system biology?
In this section we aim to cover some of the existing systems for building
interactive data exploration applications in systems biology. 

% TODO: Add related work where somebody has built a system for exploring some
% data where 

% Det er en utfordring å ha related work tidlig i dette paperet. Slik det er nå,
% er det ikke sikkert leseren skjønner hvorfor diss systemene diskuteres, hva
% som er sammenhengen mellom de, og hva det har å gjrøe med Kvik.

\subsection*{Integrate Statistical Analyses} 
OpenCPU is a system for embedded scientific computing and reproducible
research.\cite{opencpu} It provides an HTTP API to the R programming language to
provide an interface with statistical methods. It enables users to make function
calls to any R package and retrieve the results in a wide variety of formats
such as json or pdf. 
Users can chose to host their own R server or use public servers, and OpenCPU
works in a single-user setting within an R session, or a multi-user setting
facilitating multiple parallel requests. This makes OpenCPU suitable
for building a service that can run statistical analyses. 
OpenCPU provides a Javascript library for interfacing with R, as well as Docker
containers for easy installation. OpenCPU has been used to build multiple
applications.\footnote{\url{opencpu.org/apps.html}}.
% Design patterns er ikke nevnt enda
In Kvik we provide a package
to interface with OpenCPU servers from the Go programming language since it
provides the same interface to execute and run statistical analyses as we do in
our own system. 

% Skjønner ikke poenget med Renjin. Dvs hva fordelen med å kjøre R kode på en
% JVM er.
Renjin is a JVM-based interpreter for the R programming language.\cite{renjin}
It targets developers who want to integrate the R interpreter in web
applications. Since it is built on top of the JVM it allows developers to write
data exploration applications that interact directly with R code, both runnin on
top of the JVM. Although Renjin supports a large number of CRAN packages it
cannot access any R package (i.e.  any package from BioConductor or CRAN)
without modification. This makes the programming effort to use Renjin as an
interface to R higher. 

Shiny is a web application framework for R\footnote{\url{shiny.rstudio.com}}
It allows developers to
build web applications in R without having to have any knowledge about HTML, CSS
or Javascript. It provides a widget library to provide more advanced Javascript
visualizations such as Leaflet for maps or threejs for WebGL-accellerated
graphics. Developers can choose to host their own web server with the user-built
Shiny Apps, or host them on public servers. Shiny forces users to implement data
exploration applications in R, limiting the functionality to the 
widgets and libraries in Shiny. 

% Synes SparkR burde nevnes

\subsubsection*{Biogo} 
Biogo is a bioinformatics library in Go. It provides functionality to analyse
genomic and metagenomc datasets in the go programming
language.\cite{Kortschak005033} Using the go programming language the developers
are able to provide high-performance parallel processing in a clean and simple
programming language. 

\subsection*{Visualization frameworks} 
Cytoscape is an open source software platform for visualizing complex
networks and integrating these with any type of attribute
data\cite{shannon2003cytoscape}. It allows for analysis and visualization in the
same platform. Users can add additional features, such as databases connections
or new layouts, through Apps. One such app is cyREST which allows external network
creation and analysis through a REST API\cite{ono2015cyrest}.
To bring the visualization and analysis
capabilities to the web the creators of Cytoscape have developed
Cytoscape.js\footnote{\url{js.cytoscapejs.org}}, a Javascript library to create
interactive graph visualizations. 


Caleydo is a framework for building applications for visualizing and exploring
biomolecular data\cite{cleydo}. Until 2014 it was a standalone tool that needed
to be downloaded, but the Caleydo team are now making the tools web-based. There
have been several applications built using Caleydo: StratomeX for exploring
stratified heterogeneous datasets for disease subtype analysis\cite{stratomex};
Pathfinder for exploring paths in large multivariate graphs\cite{pathfinder};
UpSet to visualize and analyse sets, their intersections and
aggregates\cite{upset}; Entourage and enRoute to explore and visualize
biological pathways \cite{entourage}\cite{enroute}; LineUp to explore rankings
of items based on a set of attributes\cite{lineup}; and Domino for exploring
subsets across multiple tabular datasets\cite{domino}. 

BioJS is an open-source JavaScript framework for biological data
visualization.\cite{gomez2013biojs} It provides a community-driven online
repository with a wide range components for visualizing biological data
contributed by the bioinformatics community. BioJS builds on
node.js\footnote{\url{nodejs.org}} providing both server-side and client-side
libraries. 


\subsection*{Biological Databases} 
Maybe some words here on how to get data out of the different biological
databases? 
