\begin{abstract} % abstract, must be under 350 words
        With the advent of high-throughput genomics, there is a growing need for
        interactive data exploration tools as the systems biology community
        collects data at an unprecedented rate.  These tools need to combine
        advanced statistical analyses, relevant prior knowledge, and interactive
        visualizations in an applicaiton with clear user interfaces.  To answer
        specific research questions tools must provide specialized user
        interfaces and visualizations. While these are application-specific, the
        underlying components of a data analysis tool can be shared and reused
        later.  Application developers can therefore compose applications of
        reusable services rather than implementing a single monolithic
        application. 
        
        Our approach for developing data exploration applications in systems
        biology builds on the microservice architecture. 
        % LAB: We have identified core services/ abstractionss through...
        % (Berkeley/Spark gjengen har alltid dette som en av sine viktige
        % contributions)
        % LAB: Noe om hvordan disse services er implementert og
        % presentert for applikasjons utvikleren...som egentlig sies i setningen
        % under. Men den kan snus rundt, dvs services...to imlement
        % applications...
        The resulting applications integrate advanced statistical software,
        up-to-date information from biological databases and modern data
        visualization libaries. 
        % LAB: Demonstrate the viability through...1 application: er ikke så
        % imponerende. Så dette bør sies på en annen måte. Se feks hvordan de
        % gjør det i "ADAM" paperet, der de bruker to apps som resultater for
        % "the generatilty for" deres approach :)
        We demonstrate the viability through the MIxT Blood-Tumor web
        application that explore and compare transcriptional profiles from blood
        and tumor samples in breast cancer patients. The approach was reused in
        two other web applications and several command-line tools. Thus our
        microservice approach building on software container technology enables
        re-use and sharing of key components between application reducing
        development, deployment and maintenance time. 
        In addition the approach facilitates scaling out parts of an application
        for better performance. 
        % LA: code quality
        % BF: Skjønner ikke helt hva som menes med code quality. 

        Our approach and reference implementation Kvik is open-sourced at
        \url{github.com/fjukstad/kvik}. The web application for exploring
        transcriptional profiles, MIxT, is available at
        \url{mixt-blood-tumor.bci.mcgill.ca} and its source code at 
        \url{github.com/fjukstad/mixt}. 

\end{abstract}

